\documentclass[a4paper,10pt,oneside]{article}
\usepackage{mamba}


\title{Mamba Image Library Coding Rules and Standards}
\author{Nicolas BEUCHER \and Serge BEUCHER}
\date\today


\begin{document}
\mambaCover 
\mambaContent

 
\begin{minipage}{0.8\textwidth}
\begin{flushleft} \large
\vspace{6cm}
\textit{standards, n.:}\\[0.5cm]
\textit{The principles we use to reject other people's code.}
\end{flushleft}
\end{minipage}
\pagebreak


\section{Introduction}

Mamba is an open-source library and any contribution is welcome. In
order to help programmers, users and would be contributors, this document
presents the basic policy ruling the development of Mamba. It it our
vision, as its creators, of what is Mamba and where we want it to go
in the future. Anyone who wish to participate (by giving ideas, suggestions
or code) should read at least the policy section \ref{cha:Policy}.

This document also provides a set of rules, guidelines and general
standards to use when developping Mamba. Its intended audience is
programmers contributing to the Mamba library. These rules are not
intended to be followed blindly but are meant to make it easier for
the various programmers to understand the code they are not directly
responsible for but still need to interact with. As a programmer for
Mamba, you should take care to fulfil these rules and ultimately, if
you have a good reason for breaking them, to correctly explain and
document your reason.

Documentation contributions and licensing aspects are also covered by
this document.


\section{Policy}
\label{cha:Policy}

As you may already know, Mamba objective is to be a fast, simple,
portable and free (as in free speech but also as in free beer, this
being the case only because no one on Earth could afford it if we
had decided to sell it) mathematical morphology library. To cover
this purpose, a basic policy was decided by its creators.


\subsection{A Mathematical Morphology library}

Mamba is a mathematical morphology library only. It basically means
that there is no convolution, Fast Fourier Transform and such in it.
Only algorithms related to mathematical morphology may be present
in it.

This policy is one of the founding aspect of Mamba. The objective
it to prevent Mamba from dispersing itself into aspects that are not
related to its original purpose. We believe that there is already
enough good libraries out there to compute images using other techniques
than mathematical morphology and thus that there is no need for Mamba
to include them.


\subsection{Simple yet Fast}

Fast and simple can appear somehow contradictory as complex algorithms
may deliver faster performances. However, what we mean here is to
make sure Mamba is simple to use. As such, we believe that it is important
not to have to wait a very long time to get the result of your algorithm.
Mamba is meant to be used in research and education where you are
not always sure of your idea. Being able to rapidly evaluate the soundness
of an algorithm is also what will make Mamba simple to use. In other
words, the faster Mamba is, the simpler it becomes to try new ideas
with it.

It seems also important for us to provide tools, inside Mamba, helping
users to \textquotedbl{}visualize\textquotedbl{} their ideas and thus
be able to easily assess their worthiness. As a result, Mamba comes
with a set of tools to fulfil this objective, the most visible being
the capability to see manipulated images in live. These tools are
a very important part of the library (equally in our eyes to the core
functions used in computations). Again, the objective is to provide
a simple framework for testing and trying new ideas in mathematical
morphology.


\subsection{Portable}

Mathematical morphology algorithms variety and diversity make it somehow
stupid to try to restrain their use to a very limited set of computers
and devices (or so we think). Thus the effort towards portability.


\subsection{... and Free}

Last but not least, as we hope Mamba will become a huge success and
help spread knowledge and use of mathematical morphology, we believe
it is important not to restrain its future users with a complex or
obscure license. Similarly, we think that a commercial license is inapropriate
for some of the intended audience of this library (students, university...)
and too complex for us to handle with others (industry, ...). Thus
we decided to license Mamba under a slightly modified version of the
X11 license (also known as the MIT license). This is one of the least
restrictive free license. Any contributor to Mamba will have to make
sure his contribution is licensed under a similar license (see \ref{cha:Licensing}).


\section{Programming}


\subsection{Languages}
\begin{itemize}
\item Mamba low level is written in C. 
\item Mamba high level is written in Python. 
\end{itemize}
This language choice is meant to cover the Mamba objective of being
a fast portable and easy to use library.

C choice for low-level answers the speed requirement and partly covers
the portability. Indeed, Mamba is meant to be portable to embedded
systems. Of course, the part written in C must be compilable on various
environmments (Windows, Unix ...) and for various processors (Pentium,
Core 2 Duo, ARM, ...).

Python choice is here to ensure easiness of use. As a high level language,
it makes it easier to develop programs and algorithms without taking
care of memory management, OS portability ... Other languages could
be used for high level but currently all the high level functionnalities
are written in Python. Of course, if you have the time and the need
to develop a high level interface in another language, feel free to
do so. However, maintaining multiple high level interfaces in different
languages may prove too much hassle so your high level interface may
not be integrated to the official Mamba distribution.

\subsection{Rules for C}

Three words can be used to sum up the philosophy of rules/standards
applying to Mamba C code.
\begin{itemize}
\item simplicity 
\item readability 
\item portability 
\end{itemize}
Simplicity means that your code must not try to answer all the problems
or to take into account all the possible situations. You should leave
the complexity to the high level interface (Where generally it is
much easier to handle the real life situations).

Readability is a vague notion. Mainly, you have to make sure your
code is understandable. Comments, coherent naming, and so on are strongly
advised. More specific rules are :
\begin{itemize}
\item Use of english in comments is mandatory.
\item Function descriptions in comments use Doxygen style (see the documentation
section below for more details regarding documentation). At least all the 
exported functions must have a description.
\item Indentations are done using 4 spaces (no tabs, they are ugly because
they messed up when changing the editor). 
\item Functions and variables that are exported (visible by the exterior)
should begin with \textquotedbl{}MB\_\textquotedbl{}. 
\end{itemize}
Portability means that you should always take care to write your program
so that it will run on the greatest number of machines and equipments.
Of course this is not always feasable (particularly if you are going
very low level). Anyway, it simply means that if, for example, you
write an algorithm using SSE instructions of modern Intel/AMD processors,
you should also include your algorithm in standard C code (even if
that means it's excruciatingly slow).

\subsection{Rules for Python}

When coding in Python, you must take great care to make your code and the
various functionnalities you are implementing as simple to use as possible.

The rules presented here aim at defining good practices in the design of new 
Mamba operators in order that these operators be as general and usable as 
possible. Although these rules are not compulsory, you are advised to follow 
them if you wish to share your work. Presently, these rules concern mainly the
use of the very important arguments, edge, structuring element (se) and grid.
The use of input and output images is also considered.

\subsubsection{General rules for simplicity and readability}

The Python API is meant to be used easily by users regardless of their
proficiency at programming (of course they need to have a basic idea of what
they are doing).

Complexity is handled in the Python interface. Most of the time, C functions
only deal with image data and their depth. The Python code is in charge of 
calling the appropriate C function depending on the image depth it is 
dealing with. Thus the same function is used for binary, greyscale and 32-bit
images in Python and corresponds to three functions in C (one for
each depth). Complexity should only be handled in C if this makes
computations go significantly faster.

As for comments, you need to document each function that is meant
to be public with docstring. You should also begin your internal functions
with a \textquotedbl{}\_\textquotedbl{} that will tell Python that
it's an internal function (the same goes for global variables).

Every non internal function must provide a docstring explaining what it does,
its arguments and its output. Keep in mind that this docstring will be
automatically extracted to build the Python/Mamba reference documentation.

\subsubsection{Rules for operators using 'grid' and/or 'se' as arguments}

The following five rules adress particularly the use of the two arguments 'grid'
(grid used by the operator, hexagonal or square) and 'se' (structuring element
possibly used by the operator). As 'se' is always defined in association with a
specific grid, possible conflicts are at stake if these two arguments are used
in an antagonistic way inside the Mamba/Python scripts defining the operators.\par

\textbf{Rule n\textdegree{} 1}

When creating an operator or wrapping a C function, make sure that the grid, edge
and structuring element that may be given as argument have a default value
available when calling the Python function thus making it easier for
interactive sessions (by reducing the number of mandatory arguments).

For example, the following C function:
\lstset{language=C} 
\begin{lstlisting} 

MB_errcode MB_InfNbb(MB_Image *src,
                     MB_Image *srcdest,
                     unsigned int nbrnum,
                     unsigned int count,
                     enum MB_grid_t grid,
                     enum MB_edgemode_t edge);
\end{lstlisting}
is wrapped in Python by the function :
\lstset{language=Python}
\begin{lstlisting} 
def infNeighbor(imIn, imInout, nb, count, grid=DEFAULT_GRID, edge=FILLED):
\end{lstlisting}

Information regarding edge and grid have default values in the Python module 
and thus do not need to be defined every time.

The same rule applies for structuring element, for example:
\begin{lstlisting} 
def dilate(imIn, imOut, n=1, se=DEFAULT_SE, edge=mamba.EMPTY):
\end{lstlisting}

\textbf{Rule n\textdegree{} 2}

If 'se' is passed as an argument in the operator, as 'se' is always associated
with a grid (for instance, SQUARE3X3 is defined on the square grid), internal
operators using a grid as argument must use this grid in their argument list. 
This 'grid' argument can be passed by means of the method getGrid() of the 
structuringElement class.\par

\emph{Example:}

buildOpen definition in module openClose.py (commented script):

\lstset{language=Python}
\begin{lstlisting}
def buildOpen(imIn, imOut, n=1, se=mC.DEFAULT_SE)
    """
    Performs an opening by reconstruction operation on image 'imIn' and puts the
    result in 'imOut'. 'n' controls the size of the opening. 'se' is passed as 
    default argument.
    """
    imWrk = mamba.imageMb(imIn)
    mamba.copy(imIn, imWrk)
    mC.erode(imIn, imOut, n, se=se)
    # 'se' is used by the erosion in the first step. You can use any structuring
    # element you wish to achieve this operation.
    mC.build(imWrk, imOut, grid=se.getGrid())
    # However, as you use a build operator in the second step and as this operator
    # depends on the grid in use, you must pass to it the grid associated with 
    # 'se'. This is done through the statement grid=se.getGrid() which assigns
    # to 'grid' the grid on which 'se' is defined.
\end{lstlisting}

\par
    
\textbf{Rule n\textdegree{} 3}

If 'grid' is used in the operator argument list, internal operators needing a 
structuring element 'se' as argument must explicitely define this 'se'. Make 
sure that 'se' is compatible with the 'grid' in use.\par 

\emph{Example:}

This example is not very useful, it is just to illustrate the rule...

\lstset{language=Python}
\begin{lstlisting}
def myOperator(imIn, imOut, grid=mamba.DEFAULT_GRID)
    """
    Performs an erosion of  'imIn' and puts the result in 'imOut'. 'grid' is
    passed as default argument. If 'grid' is hexagonal, the erosion must use an 
    hexagon and a square if 'grid' is square.
    """
    if grid == mamba.HEXAGONAL:
        se = mC.HEXAGON
    else:
        se = mC.SQUARE3X3
    # The structuring element 'se' is defined according to the grid in use.
    erode(imIn, imOut, se=se)
\end{lstlisting}

If you do not define explicitely the structuring element used by the erosion, 
DEFAULT\_SE will be used. However, you have no idea of  the status of this 
default structuring element (it could be DIAMOND for instance whereas the grid 
in use is hexagonal....).

You may argue that the definition of HEXAGON or SQUARE3X3 may also be changed by 
the user. This is true and this is the reason why it is not wise to modify the 
definition of standard structuring elements.

A lot of operators use what is called ``an elementary structuring element'' 
which is in fact the size 1 structuring element defined on the grid: if the grid 
is hexagonal, it corresponds to HEXAGON (in the standard definition) and to 
SQUARE3X3 if we deal with a square grid. In this situation, if you want to be 
sure to use an unmodified structuring element, you may proceed this way instead:

\lstset{language=Python}
\begin{lstlisting}
def myOperator(imIn, imOut, grid=mamba.DEFAULT_GRID)
    """
    Performs an erosion of  'imIn' and puts the result in 'imOut'. 'grid' is
    passed as default argument. If 'grid' is hexagonal, the erosion must use an 
    hexagon and a square if 'grid' is square.
    """
    se = structuringElement(mamba.getDirections(grid), grid):
    # The structuring element 'se' is  still defined according to the grid in 
    # use. Its definition uses directly the characteristics of the grid, that is
    # the list of all its directions (including direction 0) and the grid itself
    # which is necessarily associated to 'se'.
    erode(imIn, imOut, se=se)
\end{lstlisting}

\par
    
\textbf{Rule n\textdegree{} 4}
 
Using at the same time 'se' and 'grid' in an operator argument list is strictly 
forbidden. This practice is at best redondant (if 'se' and 'grid' are 
compatible), at worst dangerous and inconsistent.\par

It is obvious that, if this rule is not enforced, this will lead sooner or 
later to a ``grid and structuring element mess''. 

\par

\textbf{Rule n\textdegree{} 5}

It is possible that no argument ('grid' or 'se') be passed to an operator. This 
is allowed provided that no conflict is generated by this means. To achieve 
this, make sure that all internal operators use only either 'grid' or 'se'. If 
it is not the case (for instance, there exists in the definition script at 
least one operator requesting 'grid' and another one requesting 'se'), this 
situation is likely to produce conflicts and errors.

If no argument is passed, the operator will use DEFAULT\_SE or DEFAULT\_GRID 
defined outside its definition space (with a possible risk of conflicts).\par

\emph{Example:}

\lstset{language=Python}
\begin{lstlisting}
def alternateFilter(imIn, imOut, n):
    """
    Performs an alternate filter operation of size 'n' on image 'imIn' and puts
    the result in 'imOut'. If 'openFirst' is True, the filter begins with an
    opening, a closing otherwise.
    """
    if openFirst:
        mC.open(imOut, imOut, n)
        mC.close(imOut, imOut, n)
    else:
        mC.close(imOut, imOut, n)
        mC.open(imOut, imOut, n)
    # Neither open nor close need 'grid' in their arguments, but only 'se'. 
    # Therefore, as there is, in the definition of alternateFilter, no other
    # operator requesting 'grid', no conflict occurs and the structuring element
    # used by open and close will be DEFAULT_SE.
\end{lstlisting}

Although this rule is admitted, it is not wise in practice to apply it. Indeed, 
to be applied without risk, this rule states that all internal operators need 
only one argument, 'grid' or (exclusive) 'se', it is therefore more efficient 
and safe to systematically pass this argument in the operator argument list. 
This avoids later difficulties if the operator itself is used in the definition 
of more complex functions.

Thus, the following definition of 'alternateFilter' is safer and usable inside 
another definition:

\lstset{language=Python}
\begin{lstlisting}
def alternateFilter(imIn, imOut,n, se=mamba.DEFAULT_SE):
    """
    Performs an alternate filter operation of size 'n' on image 'imIn' and puts
    the result in 'imOut'. If 'openFirst' is True, the filter begins with an
    opening, a closing otherwise.
    """
    if openFirst:
        mC.open(imOut, imOut, n, se=se)
        mC.close(imOut, imOut, n, se=se)
    else:
        mC.close(imOut, imOut, n, se=se)
        mC.open(imOut, imOut, n, se=se)
\end{lstlisting}

These five rules are not mandatory and do not constitute a dogma. However, 
using them will avoid some nasty problems during your first steps with Mamba. 
They allow also a better applications sharing.

Most of the time, 'grid' and 'se' are passed in the argument list as default 
arguments (se=mC.DEFAULT\_SE and grid=mamba.DEFAULT\_GRID). This is very handy 
when you use Mamba operators in terminal/console mode to test ideas and to 
concatenate these operators in order to solve a problem. In this case, you 
generally work in a chosen environment, you know which grid you are using (the 
grid which is defined as default grid) and you control totally your structuring 
elements. For instance, if your grid is hexagonal, it is likely that your 
default structuring element is set to HEXAGON.

This interesting simplification when in interactive mode, is unacceptable when 
you define a new operator if you want to share it with other users. In this 
case, you do not control the environment anymore and you must make sure that 
this operator will be used according to your wishes. Therefore, it is of the 
outmost importance to correctly manage the 'grid' and 'se' arguments in your 
definition. 
 
\subsubsection{Rules for input and output images}

You can define with Mamba new operators using one or more input image(s) and 
putting some results in one or more output image(s). An important programming 
rule is that your operators should be defined in such a way that the same image 
may be used both as input and output. This means that you must be aware to 
avoid to replace your input image by any output image prematurely. Indeed, if 
you do not take care of this rule, it is likely that, in most cases, no error 
will occur. However, your result will certainly be wrong.\par

\emph{Example:}

\lstset{language=Python}
\begin{lstlisting}
# First approach (bad one).

def myOper(imIn, imOut):
    """
    This operator computes  a very simple morphological gradient of imIn and
    puts the result in imOut.
    """
    imWrk = imageMb(imIn)
    dilate(imIn, imOut)
    erode(imIn, imWrk)
    sub(imOut, imWrk, imOut)
\end{lstlisting}

In this example, if imIn and imOut are identical, although no error is detected, 
the final result will be incorrect since, in the erode operator, imIn has been 
replaced by its dilation.

\lstset{language=Python}
\begin{lstlisting}
# Correct approach.

def myOper(imIn, imOut):
    """
    This operator computes  very simple morphological gradient of imIn and
    puts the result in imOut.
    """
    imWrk = imageMb(imIn)
    dilate(imIn, imWrk)
    erode(imIn, imOut)
    sub(imWrk, imOut, imOut)
\end{lstlisting}

In this case, imIn may possibly be overwritten after the erosion with no harmful 
consequences regarding the final result.

Here again, enforcing this rule makes your operators user friendly and their 
sharing is made easier. This rule can be implemented very easily by using 
intermediary working images in your definition and by copying the input imIn 
into the working image at the beginning of this definition. 


\subsubsection{Rules for 'edge' argument in user-defined operators}

Contrary to 'grid' and 'se', it is generally neither necessary nor wise  to 
pass 'edge' in your operators arguments list. 

There is no default value in Mamba for 'edge'. The reason of this is that 'edge' 
is always used to impose specific behavior and properties to some operators. 
It is the case for instance with geodesic operators.  It is well known that non 
extensive geodesic operators need that 'edge' be defined as 'FILLED'. Obviously, 
for these operators, the edge setting is defined inside the operator definition 
and not at all in its argument list.

In fact, 'edge' allows to define two different contexts for your transformation 
depending on the way you consider the outer space, that is the space outside 
your image window.

If you consider that the outer space is empty, you are then in an euclidean 
context. In this configuration, 'edge' is defined as 'EMPTY' and the main 
consequence of this choice is that outer 'black' pixels may have some influence 
on the result of your transformation for inside pixels, in particular if your 
transformation is anti-extensive. Note that you could equally define the outer 
space as 'FILLED' in this context (and Mamba allows it without problem). 
However, this configuration is seldom used in practice, being considered as 
``unnatural'' (we prefer to imagine the outer space completely empty rather 
than totally filled!).

If you consider that you have no idea of the content of the outer space and, 
moreover, that you don't care of it, you are working in a geodesic context. 
Your image window is considered as a geodesic space and you do not have to take 
the outside pixels into account in your operator. This can be achieved by 
setting 'edge' accordingly in your definition. If  an extensive operator 
requesting the setting of 'edge' is called in the definition, 'edge' should be 
set to 'EMPTY'. Conversely, if the operator is anti-extensive, 'edge' should be 
set to 'FILLED'.

It is important to keep in mind the fact that an empty edge is NOT equivalent 
to an euclidean context and, conversely, that a filled edge is NOT equivalent 
to a geodesic context. This is the main reason why 'edge' should not be passed 
in your operator argument list.

The only case where this rule can be by-passed is when you know that the 
operator you are defining can be used in both contexts and that it produces a 
legitimate result (but possibly different). This is true for instance for the 
basic morphological operators: dilations, erosions, openings, closings, 
thinnings and thickenings. So, when defining another implementation of these 
basic transforms (this is perfectly allowed), it is nevertheless wise (if not 
compulsory) to pass 'edge' in the operator argument list.

\section{Documentation}

\subsection{In code}

As was explained in the programming section, documentation related
to code must take two forms :

For Python code, use docstring.

For C code, use Doxygen style.

You should indeed know that part of the actual documentation is created
automatically using these forms of documentation. The whole documentation
must be written in english.

\subsection{Other documents}

All the other documents existing to date were created using Latex.
Mamba comes with a specific Latex style that allows creating an homogeneous
set of documents (with header, code listing style, etc... coherent
between documents).

If you have an idea of documentation, we strongly advise you to use
Latex along with the Mamba style (which can be found in the source).
To use it, you simply need to create a directory texmf/tex/latex in
your \$HOME or wherever your Latex distribution may find it and add
the files mamba.sty and mamba\_logo\_white.png in it. We prefer Latex
because it makes it easier to track modifications inside our versioning
repository (Latex being text format and not binary).

For those of you who do not have Latex or who do not wish to use it,
we can accept documents in other formats provided you also give along
a PDF version of them (making sure that way that the document will
be readable by anyone).

And of course, whatever the format you choose, make sure your name
appears in the document.

\section{Testing}

Mamba is tested to ensure that it works properly and that it
implements correctly the mathematical morphology algorithms it uses.

Testing is performed in a specific environment. We tried to avoid verification
with test images because they are not easy to handle (often a lot of images to
store) and although they correctly reveal errors, they cannot be used to
identify and correct them. Another problem with test images is the production
of the expected result image and the insurance that it is correct.

There are two levels of testing:

\begin{itemize}
\item \textbf{\textsc{Basic C level operators/functions}} :
They are tested and verified as extensively as possible to ensure that they
are working properly and do not crash. The testing is performed by calling their
wrapping Python function. Algorithm soundness, edge effects,
grid effects, image depth acceptation, proper error handling are tested for every
C function.
\item \textbf{\textsc{Mathematical operators and other high-level Python functions}} :
These are tested less thoroughly as this is not a realistic task. Basic
algorithmic verification is performed. The idea is to cover all the Python code
(100\% code coverage) to make sure there is no typo.
\end{itemize}

Although we take great care of producing a bug free library, there is no way
for us to guarantee that. Tests are likely to have some blind spot. They are
mainly here to ensure a bottom line quality level and to prevent any regression.
If you have written some operators/functions and would like to add it into the
official Mamba release, we would be grateful if you could provide us with some
tests to verify them.

\section{Licensing}

\label{cha:Licensing}

For code contribution, make sure your work is licensed under a similar
license than the one covering Mamba. Here is a reminder of the license:

%\begin{minipage}{0.8\textwidth}
%\small
%Copyright (c) <2009>, <Your name here>
%\vspace{0.5cm}
%Permission is hereby granted, free of charge, to any person
%obtaining a copy of this software and associated documentation files
%(the \textquotedbl{}Software\textquotedbl{}), to deal in the Software
%without restriction, including without limitation the rights to use,
%copy, modify, merge, publish, distribute, sublicense, and/or sell
%copies of the Software, and to permit persons to whom the Software
%is furnished to do so, subject to the following conditions: The above
%copyright notice and this permission notice shall be included in all
%copies or substantial portions of the Software.
%\vspace{0.5cm}
%THE SOFTWARE IS PROVIDED \textquotedbl{}AS IS\textquotedbl{},
%WITHOUT WARRANTY OF ANY KIND, EXPRESS OR IMPLIED, INCLUDING BUT NOT
%LIMITED TO THE WARRANTIES OF MERCHANTABILITY, FITNESS FOR A PARTICULAR
%PURPOSE AND NONINFRINGEMENT. IN NO EVENT SHALL THE AUTHORS OR COPYRIGHT
%HOLDERS BE LIABLE FOR ANY CLAIM, DAMAGES OR OTHER LIABILITY, WHETHER
%IN AN ACTION OF CONTRACT, TORT OR OTHERWISE, ARISING FROM, OUT OF
%OR IN CONNECTION WITH THE SOFTWARE OR THE USE OR OTHER DEALINGS IN
%THE SOFTWARE.
%\vspace{1cm}
%\end{minipage}

%
\begin{minipage}[c]{0.8\textwidth}%
 {\small Copyright (c) <2009>, <Your name here>}{\small \vspace{0.5cm} \par}

{\small Permission is hereby granted, free of charge, to any person
obtaining a copy of this software and associated documentation files
(the \textquotedbl{}Software\textquotedbl{}), to deal in the Software
without restriction, including without limitation the rights to use,
copy, modify, merge, publish, distribute, sublicense, and/or sell
copies of the Software, and to permit persons to whom the Software
is furnished to do so, subject to the following conditions: The above
copyright notice and this permission notice shall be included in all
copies or substantial portions of the Software.}{\small \vspace{0.5cm} \par}

{\small Except as contained in this notice, the names of the above copyright 
holders shall not be used in advertising or otherwise to promote the sale, use 
or other dealings in this Software without their prior written authorization.}
{\small \vspace{0.5cm} \par}

{\small THE SOFTWARE IS PROVIDED \textquotedbl{}AS IS\textquotedbl{},
WITHOUT WARRANTY OF ANY KIND, EXPRESS OR IMPLIED, INCLUDING BUT NOT
LIMITED TO THE WARRANTIES OF MERCHANTABILITY, FITNESS FOR A PARTICULAR
PURPOSE AND NONINFRINGEMENT. IN NO EVENT SHALL THE AUTHORS OR COPYRIGHT
HOLDERS BE LIABLE FOR ANY CLAIM, DAMAGES OR OTHER LIABILITY, WHETHER
IN AN ACTION OF CONTRACT, TORT OR OTHERWISE, ARISING FROM, OUT OF
OR IN CONNECTION WITH THE SOFTWARE OR THE USE OR OTHER DEALINGS IN
THE SOFTWARE. }%
\vspace{1cm}
\end{minipage}

Any similar license is appropriate (see X11 license, BSD license). Make sure
your license is compatible with GNU GPL and that it has NO copyleft
obligations (GPL is a copyleft license). If your license has a copyleft
obligation, we will not be able to add your code to the Mamba source. However
an optional package/module can be created that will let the user decide if the
copyleft obligation is a problem for her/him or not (see the mambaRealtime for
an example of this situation).

For documentation contributions, make sure that your work license
allow us to distribute it freely.

\section{Other contributions}

There are lot of things you could do for Mamba, even if your are not
a programmer or a writer.

First of all, you can give us feedback regarding the way you use Mamba.
Any comment, criticism or suggestion is welcome and will be taken
into consideration (as long as it does not infringe our policy).

\end{document}

